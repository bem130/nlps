\documentclass[a4paper,10.5pt]{ltjsarticle}
\usepackage[margin=15truemm]{geometry}
\usepackage{url}
\usepackage{listings, jvlisting, color, graphicx}
\usepackage{plantuml}
\usepackage{tikz}

\setcounter{secnumdepth}{6}
\lstset{
    frame={tb},
    breaklines=true,
    backgroundcolor={\color[rgb]{0.95,1,1}},
    columns=[l]{fullflexible},
    xrightmargin=0\zw,
    xleftmargin=4\zw,
}
\begin{document}
\begin{titlepage}

\title{\Huge Neknaj Language Processing System}
\author{\LARGE Bem130}
\date{\number\year \slash \number\month \slash \number\day}

\maketitle


\tableofcontents

\end{titlepage}




\part{概要}
スタックマシンを基本にしたBem130の自作プログラミング言語とその処理システム

% \begin{center}
%     \begin{tikzpicture}
%         \node (NLP) {NLP};
%         \node (NLPjs) {NLP.js};
%         \node (NVM) {NVM};

%         \path[->, >=stealth]
%         (NLP) edge[left, bend right=10] node{0.2} (NLPjs);
%     \end{tikzpicture}
% \end{center}



\section{特徴}
\begin{table}[h]
    \centering
    \begin{tabular}{lll}
        \hline
        特徴 & 理由 & 主な対象 \\
        \hline \hline
        逆ポーランド記法 & \begin{tabular}{ll} 中置演算子や括弧を含む式の解析が難しかった為\\引数の式を先に書くことで処理の順番が明確になる為 \end{tabular}& NLP \\
        \hline
        代入を表す:> & \begin{tabular}{ll} 等号として用いられる=との違いを明確にするため\\代入の方向を明確にする為\\顔文字のようで可愛い為 \end{tabular} & NLP \\
        \hline
        右に記述する代入先の変数 & 式を先に書くことで代入の処理の順番が明確になる為 & NLP \\
        \hline
        コメントアウトとノート & 不要なコードと、必要なメモを区別するため & NLP \\
        \hline
        関数の定義の巻き上げ & 定義文の前でも使用できるのが便利で気に入った為 & NLP \\
        変数の定義の巻き上げ & 同じスコープの名前が指すものを統一する為 & NLP \\
        浮動小数点数は基数10が基本 & 2進化による丸め誤差が気に入らなかった為 & BemLib for NVM \\
        コンパイル結果をincludeする & ソースコードのincludeが面倒そうに感じた為 & NLPS \\
        むやみにハンドリングする例外 & 例外の為に特別な処理を作るのが気に入らなかった為 & NLPS \\
        \hline
    \end{tabular}
\end{table}

\part{B-debt --- Nekanj Programming Language}
逆ポーランド記法を基本とするプログラミング言語
\section{名称}
当初NLP(Neknaj Language for Programming)としていたが、NML(Neknaj Markup Language)との統一や、NLP(Natural Language for Programming,自然言語処理)との混同防止の為、NPL(Neknaj Programming Language)と改名した。\\
また、NPLへの改名によって、NPL(Non-Performing Loan,不良債権)と被った為、英語愛称を「Bdebt」「B-debt」(bad debtより)、日本語愛称を「不債」とする。

\section{サンプルコード}
\begin{lstlisting}[]
!include: stdcalc;
!using: stdcalc;
!replace: pi: 3.1415; #* this is a block comment *#

!fn: 4.int(4.int: max): main {
    !local: 4.int: z;  # this is a line comment
    0 0 add :> !local: 4.int: y;
    0 :> return;
}
\end{lstlisting}

\section{ツールキット}
JavaScript版とC++版があるが、どちらも完成していない
\begin{table}[h]
    \centering
    \begin{tabular}{lcllllll}
        \hline
        種類 & ファイル名 & 説明 \\
        \hline \hline
        仮想マシン & nve.worker.js \\
        仮想マシン & nve.worker.cpp \\
        \hline
        コンパイラ類 & nlp.ts \\
        コンパイラ類 & nlp.js & nlp.tsをコンパイルしたもの \\
        \hline
        エディタ & editor.html & nlp.ts向けのGUI \\
        エディタ & debugger.html & nlp.ts向けのGUI, editor.htmlよりも多くの情報を表示 \\
        \hline
    \end{tabular}
\end{table}


\section{記法}

\subsection{コメント}
\begin{tikzpicture}[leaf/.style={rectangle,draw},level 1/.style={level distance=30mm}]
    \node {$コメント$} [grow=right]
    child { node {$コメントアウト$}
        child { node[leaf] {$インラインコメント$} }
        child { node[leaf] {$ブロックコメント$} }
    }
    child { node[leaf] {$ノート$}
    };
\end{tikzpicture}
\begin{lstlisting}[]
#: ノート
\end{lstlisting}
\begin{lstlisting}[]
# インラインコメント
\end{lstlisting}
\begin{lstlisting}[]
#* ブロックコメント *#
\end{lstlisting}
\subsection{型}
\begin{lstlisting}[]
サイズ.種類
\end{lstlisting}

\subsection{式}
トークンをスペースでつないだもの
\begin{lstlisting}[]
トークン トークン トークン ...
\end{lstlisting}

\subsection{文}
    \subsubsection{トップレベルの文}
        \paragraph{include}
\begin{lstlisting}[]
!include: 名前;
\end{lstlisting}
        \paragraph{using}
\begin{lstlisting}[]
!using: 名前;
\end{lstlisting}
        \paragraph{global}
\begin{lstlisting}[]
!global: 型: 名前;
\end{lstlisting}
        \paragraph{replace}
\begin{lstlisting}[]
!replace: 置き換え前: 置き換え後;
\end{lstlisting}

    \subsubsection{ブロック内の文}
        \paragraph{式文}
\begin{lstlisting}[]
式;
\end{lstlisting}
        \paragraph{local宣言}
\begin{lstlisting}[]
!local: 型: 名前;
\end{lstlisting}
        \paragraph{代入}
\begin{lstlisting}[]
式 :> 名前;
\end{lstlisting}
        \paragraph{local宣言付代入}
\begin{lstlisting}[]
式 :> !local: 型: 名前;
\end{lstlisting}
        \paragraph{return}
\begin{lstlisting}[]
式 :> return;
\end{lstlisting}


\subsection{ブロック}
文,構造を複数書いたもの\\
構造には、単ブロックと制御構造が含まれる\\
ブロック要素には、文と構造が含まれる

ブロックは入れ子にすることができる

\begin{lstlisting}[]
{
    ブロック要素
    ブロック要素
    ブロック要素
    ...
}
\end{lstlisting}

\subsection{単ブロック}
\begin{lstlisting}[]
ブロック;
\end{lstlisting}

\subsection{制御構造}
\subsubsection{基本形}
条件式,種類,ブロック を組にしたもの\\
種類によって細かい記法は異なる
\begin{lstlisting}[]
!ctrl:(式) 種類 ブロック;
\end{lstlisting}
\subsubsection{while}
\begin{lstlisting}[]
!ctrl:(式) while ブロック;
\end{lstlisting}
\subsubsection{if}
基本となるifの後ろに、任意個のelseifと一つのelseを付けることができる
\begin{lstlisting}[]
!ctrl:(式) if ブロック;
\end{lstlisting}
\begin{lstlisting}[]
!ctrl:(式) if ブロック else ブロック;
\end{lstlisting}
\begin{lstlisting}[]
!ctrl:(式) if ブロック (式) elseif ブロック;
\end{lstlisting}
\begin{lstlisting}[]
!ctrl:(式) if ブロック (式) elseif ブロック else ブロック;
\end{lstlisting}

\subsection{関数}

\begin{lstlisting}[]
!fn:戻り値型(引数): 名前 ブロック;
\end{lstlisting}

\subsubsection{引数}
\begin{lstlisting}[]
型: 名前, 型: 名前, 型: 名前, ...;
\end{lstlisting}

\section{Nekanj Virtual Machine}
\subsection{概要}
1ワード32bits(4Bytes)のスタックマシン

\subsection{命令セット}
\begin{table}[h]
    \centering
    \begin{tabular}{lcllllll}
        \hline
        命令 & 引数 & 消費 & 追加 & スタック長 & 処理\\
        \hline \hline
        00 & push & v & - & v & +1 & スタックに値vを入れる\\
        01 & fram & n & - & 0(\times n) & +n & スタックにn回0を入れる\\
        02 & pop & - & v & - & -1 & スタックトップの値vを1つ消す\\
        03 & popn & n &v(\times n) - &  & -n & スタックトップの値vをn個消す\\
        \hline
        04 & setv & l & v & - & -1 & l個目のローカル変数に値vを入れる\\
        05 & getv & l & - & v & +1 & l個目のローカル変数から値vを複製する\\
        06 & setgv & g & v & - & -1 & g個目のグローバル変数に値vを入れる\\
        07 & getgv & g & - & v & +1 & g個目のグローバル変数から値vを複製する\\
        08 & seth & - & h v & - & -2 & ヒープ領域のh番目に値vを入れる\\
        09 & getv & - & h & v & +1 & ヒープ領域のh番目から値vを複製する\\
        \hline
        0a & jmp & p & - & - & ±0 & アドレスpまでジャンプする \\
        0b & ifjmp & p & cn & - & -1 & cnがtrueならば、アドレスpまでジャンプする \\
        0c & call & p & - & fp pc & +2 & 関数をの呼ぶ処理をし、アドレスpまでジャンプする \\
        0d & ret & n & fp pc  v(\times n) & - & -2-n & 関数を呼ぶ前に戻って、引数分n回popする \\
        \hline
        10 & equ & - & a b & v & -1 & a == b \\
        11 & les & - & a b & v & -1 & a < b \\
        12 & grt & - & a b & v & -1 & a > b \\
        \hline
        13 & not & - & a & v & ±0 & not a \\
        14 & and & - & a b & v & -1 & a and b \\
        15 & or & - & a b & v & -1 & a or b \\
        16 & xor & - & a b & v & -1 & a xor b \\
        \hline
        17 & notb & - & a & v & ±0 & not a \\
        18 & andb & - & a b & v & -1 & a and b \\
        19 & orb & - & a b & v & -1 & a or b \\
        1a & xorb & - & a b & v & -1 & a xor b \\
        1b & lsft & - & a b & v & -1 & a << b \\
        1b & rsft & - & a b & v & -1 & a >> b \\
        \hline
        20 & add & - & a b & v & -1 & a + b \\
        21 & addc & - & a b x & c s & -1 & a + b 繰り上がりはc \\
        \hline
    \end{tabular}
\end{table}



\begin{table}[]
    \centering
    \caption{略語}
    \begin{tabular}{|c|l|}
        \hline
        NLPS & Neknaj Language Processing System\\
        NPL & Neknaj Programming Language\\
        NLPO & Neknaj Language for Programming - Object file\\
        NVA, NVASM & Neknaj Virtual machine - Assembly language\\
        NVMC & Neknaj Virtual machine - Machine Code\\
        NVM & Neknaj Virtual Machine\\
        \hline
    \end{tabular}
\end{table}


\end{document}