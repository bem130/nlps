\section{特徴}
\begin{table}[h]
    \centering
    \begin{tabular}{lll}
        \hline
        特徴 & 理由 & 主な対象 \\
        \hline \hline
        逆ポーランド記法 & \begin{tabular}{ll} 中置演算子や括弧を含む式の解析が難しかった為\\引数の式を先に書くことで処理の順番が明確になる為 \end{tabular}& Bdebt \\
        \hline
        代入を表す:> & \begin{tabular}{ll} 等号として用いられる=との違いを明確にするため\\代入の方向を明確にする為\\顔文字のようで可愛い為 \end{tabular} & Bdebt \\
        \hline
        右に記述する代入先の変数 & 式を先に書くことで代入の処理の順番が明確になる為 & Bdebt \\
        \hline
        コメントアウトとノート & 不要なコードと、必要なメモを区別するため & Bdebt \\
        \hline
        関数の定義の巻き上げ & 定義文の前でも使用できるのが便利で気に入った為 & Bdebt \\
        変数の定義の巻き上げ & 同じスコープの名前が指すものを統一する為 & Bdebt \\
        浮動小数点数は基数10が基本 & 2進化による丸め誤差が気に入らなかった為 & BemLib for NVM \\
        コンパイル結果をincludeする & ソースコードのincludeが面倒そうに感じた為 & NLPS \\
        むやみにハンドリングする例外 & 例外の為に特別な処理を作るのが気に入らなかった為 & NLPS \\
        \hline
    \end{tabular}
\end{table}