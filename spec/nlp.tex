\part{Nekanj Language for Programming}

逆ポーランド記法を基本とするプログラミング言語

\section{サンプルコード}
\begin{lstlisting}[]
    !include: stdcalc;
    !using: stdcalc;
    !replace: pi: 3.1415;
    #* block comment *#

    !fn: 4.int(4.int: max): main {
        !local: 4.int: z;  # this is a line comment
        0 0 add :> !local: 4.int: y;
        0 :> return;
    }
\end{lstlisting}

\section{ツールキット}
JavaScript版とC++版があるが、どちらも完成していない
\begin{table}[h]
    \centering
    \begin{tabular}{lcllllll}
        \hline
        種類 & ファイル名 & 説明 \\
        \hline \hline
        仮想マシン & nve.worker.js \\
        仮想マシン & nve.worker.cpp \\
        \hline
        コンパイラ類 & nlp.ts \\
        コンパイラ類 & nlp.js & nlp.tsをコンパイルしたもの \\
        \hline
        エディタ & editor.html & nlp.ts向けのGUI \\
        エディタ & debugger.html & nlp.ts向けのGUI, editor.htmlよりも多くの情報を表示 \\
        \hline
    \end{tabular}
\end{table}


\section{記法}

\subsection{コメント}
\begin{tikzpicture}[leaf/.style={rectangle,draw},level 1/.style={level distance=30mm}]
    \node {$コメント$} [grow=right]
    child { node {$コメントアウト$}
        child { node[leaf] {$インラインコメント$} }
        child { node[leaf] {$ブロックコメント$} }
    }
    child { node[leaf] {$ノート$}
    };
\end{tikzpicture}
\begin{lstlisting}[]
#: ノート
# インラインコメント
#* ブロックコメント *#
\end{lstlisting}

\subsection{文}

\subsection{式}

\subsection{型}