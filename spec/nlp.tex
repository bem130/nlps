\part{Nekanj Language for Programming}

逆ポーランド記法を基本とするプログラミング言語

\section{サンプルコード}
\begin{lstlisting}[]
!include: stdcalc;
!using: stdcalc;
!replace: pi: 3.1415;
#* block comment *#

!fn: 4.int(4.int: max): main {
    !local: 4.int: z;  # this is a line comment
    0 0 add :> !local: 4.int: y;
    0 :> return;
}
\end{lstlisting}

\section{ツールキット}
JavaScript版とC++版があるが、どちらも完成していない
\begin{table}[h]
    \centering
    \begin{tabular}{lcllllll}
        \hline
        種類 & ファイル名 & 説明 \\
        \hline \hline
        仮想マシン & nve.worker.js \\
        仮想マシン & nve.worker.cpp \\
        \hline
        コンパイラ類 & nlp.ts \\
        コンパイラ類 & nlp.js & nlp.tsをコンパイルしたもの \\
        \hline
        エディタ & editor.html & nlp.ts向けのGUI \\
        エディタ & debugger.html & nlp.ts向けのGUI, editor.htmlよりも多くの情報を表示 \\
        \hline
    \end{tabular}
\end{table}


\section{記法}

\subsection{コメント}
\begin{tikzpicture}[leaf/.style={rectangle,draw},level 1/.style={level distance=30mm}]
    \node {$コメント$} [grow=right]
    child { node {$コメントアウト$}
        child { node[leaf] {$インラインコメント$} }
        child { node[leaf] {$ブロックコメント$} }
    }
    child { node[leaf] {$ノート$}
    };
\end{tikzpicture}
\begin{lstlisting}[]
#: ノート
\end{lstlisting}
\begin{lstlisting}[]
# インラインコメント
\end{lstlisting}
\begin{lstlisting}[]
#* ブロックコメント *#
\end{lstlisting}
\subsection{型}
\begin{lstlisting}[]
サイズ.種類
\end{lstlisting}

\subsection{式}
トークンをスペースでつないだもの
\begin{lstlisting}[]
トークン トークン トークン ...
\end{lstlisting}

\subsection{文}
    \subsubsection{トップレベルの文}
        \paragraph{include}
\begin{lstlisting}[]
!include: 名前;
\end{lstlisting}
        \paragraph{using}
\begin{lstlisting}[]
!using: 名前;
\end{lstlisting}
        \paragraph{global}
\begin{lstlisting}[]
!global: 型: 名前;
\end{lstlisting}
        \paragraph{replace}
\begin{lstlisting}[]
!replace: 置き換え前: 置き換え後;
\end{lstlisting}

    \subsubsection{ブロック内の文}
        \paragraph{式文}
\begin{lstlisting}[]
式;
\end{lstlisting}
        \paragraph{local宣言}
\begin{lstlisting}[]
!local: 型: 名前;
\end{lstlisting}
        \paragraph{代入}
\begin{lstlisting}[]
式 :> 名前;
\end{lstlisting}
        \paragraph{local宣言付代入}
\begin{lstlisting}[]
式 :> !local: 型: 名前;
\end{lstlisting}
        \paragraph{return}
\begin{lstlisting}[]
式 :> return;
\end{lstlisting}


\subsection{ブロック}
文,構造を複数書いたもの\\
構造には、単ブロックと制御構造が含まれる\\
ブロック要素には、文と構造が含まれる

ブロックは入れ子にすることができる

\begin{lstlisting}[]
{
    ブロック要素
    ブロック要素
    ブロック要素
    ...
}
\end{lstlisting}

\subsection{単ブロック}
\begin{lstlisting}[]
ブロック;
\end{lstlisting}

\subsection{制御構造}
\subsubsection{基本形}
条件式,種類,ブロック を組にしたもの\\
種類によって細かい記法は異なる
\begin{lstlisting}[]
!ctrl:(式) 種類 ブロック;
\end{lstlisting}
\subsubsection{while}
\begin{lstlisting}[]
!ctrl:(式) while ブロック;
\end{lstlisting}
\subsubsection{if}
基本となるifの後ろに、任意個のelseifと一つのelseを付けることができる
\begin{lstlisting}[]
!ctrl:(式) if ブロック;
\end{lstlisting}
\begin{lstlisting}[]
!ctrl:(式) if ブロック else ブロック;
\end{lstlisting}
\begin{lstlisting}[]
!ctrl:(式) if ブロック (式) elseif ブロック;
\end{lstlisting}
\begin{lstlisting}[]
!ctrl:(式) if ブロック (式) elseif ブロック else ブロック;
\end{lstlisting}

\subsection{関数}

\begin{lstlisting}[]
!fn:戻り値型(引数): 名前 ブロック;
\end{lstlisting}

\subsubsection{引数}
\begin{lstlisting}[]
型: 名前, 型: 名前, 型: 名前, ...;
\end{lstlisting}